% --------------------------------------------------------------
% This is all preamble stuff that you don't have to worry about.
% Head down to where it says "Start here"
% --------------------------------------------------------------
 
\documentclass[12pt]{article}
 
\usepackage[margin=1in]{geometry} 
\usepackage{amsmath,amsthm,amssymb}
\usepackage{enumerate} 

\newenvironment{proposition}[2][Proposition]{\begin{trivlist}
\item[\hskip \labelsep {\bfseries #1}\hskip \labelsep {\bfseries #2.}]}{\end{trivlist}}

%% set noindent by default and define indent to be the standard indent length
\newlength\tindent
\setlength{\tindent}{\parindent}
\setlength{\parindent}{0pt}
\renewcommand{\indent}{\hspace*{\tindent}}

\begin{document}
 
% --------------------------------------------------------------
%                         Start here
% --------------------------------------------------------------
 
\title{Assignment 2}
\author{David Fleischer -- 27101449\\ 
MACF 402 - Mathematical \& Computational Finance II}
 
\maketitle

%=========Problem 1===========%
{\bf Problem 1}.

\begin{align*}
	\mathbb E[I(T)] &= \mathbb E\Big[\sum^{n-1}_{i=0} H_{t_i}\big(B_{t_{i+1}} - B_{t_i}\big)\Big] \\
	&= \sum^{n-1}_{i=0} \mathbb E\Big[H_{t_i}\big(B_{t_{i+1}} - B_{t_i}\big)\Big] \quad \text{(by linearity of expectation)}\\
	&= \sum^{n-1}_{i=0} \mathbb E\Big[\mathbb E\big[ H_{t_i}\big(B_{t_{i+1}} - B_{t_i}\big)| \mathcal F_{t_i} \big]\Big] \quad \text{(by the tower property)} \\
	&= \sum^{n-1}_{i=0} \mathbb E\Big[H_{t_i} \mathbb E\big[\big(B_{t_{i+1}} - B_{t_i}\big)| \mathcal F_{t_i} \big]\Big] \quad \text{(since $H_{t_i}$ is $\mathcal F_{t_i}$-measurable)} \\
	&= \sum^{n-1}_{i=0} \mathbb E\Big[H_{t_i} \big(\mathbb E\big[B_{t_{i+1}} \big| \mathcal F_{t_i}\big] - \mathbb E\big[B_{t_i}\big| \mathcal F_{t_i} \big]\big)\Big]  \quad \text{(by linearity of conditional expectation)} \\
	&= \sum^{n-1}_{i=0} \mathbb E\Big[H_{t_i} \big(\mathbb E\big[B_{t_{i+1}} \big| \mathcal F_{t_i}\big] - \mathbb E\big[B_{t_i}\big]\big)\Big]  \quad \text{(since $B_{t_i}$ is $\mathcal F_{t_i}$-measurable)} \\
	&= \sum^{n-1}_{i=0} \mathbb E\Big[H_{t_i} \big(\mathbb E\big[B_{t_{i}}\big] - \mathbb E\big[B_{t_i}\big]\big)\Big]  \quad \text{(since $B_{t_i}$ is a martingale)} \\
	&= \sum^{n-1}_{i=0} \mathbb E\Big[H_{t_i} \cdot 0\Big] = 0
\end{align*}


%=========Problem 2===========%
{\bf Problem 2}.

\indent Since we are looking for the SDE satisfied by $\ln S_t$ we consider $f(x) = \ln x$. Computing our derivatives:
\begin{equation*}
	f_x(x) = \frac{1}{x} \quad f_{xx}(x) = -\frac{1}{x^2}
\end{equation*}

and so from It\^{o}'s formula we have
\begin{equation*}
	f(x) = f(0) + \int^t_0 \frac{1}{x} \,dx - \frac{1}{2}\int^t_0 \frac{1}{x^2} \,d\langle x_{(\cdot)}\rangle_u
\end{equation*}

Evaluating for $S_t$ gives us
\begin{equation*}
	\ln S_t = S_0 + \int^t_0 \frac{1}{S_u} \,dS_u - \frac{1}{2}\int^t_0 \frac{1}{S_u^2} \,d\langle S_{(\cdot)}\rangle_u
\end{equation*}

We have
\begin{equation*}
	dS_t = rS_t\,dt + \sigma S_t\,dW_t
\end{equation*}

Thus
\begin{equation*}
	\,d\langle S_{(\cdot)}\rangle_t = (\sigma S_t)^2\,dt
\end{equation*}

and so
\begin{align*}
	\ln S_t &= S_0 + \int^t_0 \frac{1}{S_u} \big[rS_u\,du + \sigma S_u\,dW_u \big] - \frac{1}{2}\int^t_0 \frac{1}{S_u^2} (\sigma S_u)^2\,du \\
	&= S_0 + \int^t_0 \big[r\,du + \sigma\,dW_u \big] - \frac{1}{2}\int^t_0 \sigma^2\,du \\
	&= S_0 + \int^t_0 \big[r - \frac{1}{2}\sigma^2\big] \,du + \int^t_0 \sigma\,dW_u
\end{align*}



%=========Problem 3===========%
{\bf Problem 3}.

%=========Problem 5===========%
{\bf Problem 5}. Consider the process $X_t$ given by the SDE
\begin{equation*}
	dX_t = -X_t\,dt + e^{-t}\,dB_t
\end{equation*}

with $X_0 = 0$ and $B_t$ standard Brownian motion. Show that
\begin{equation*}
	\mathbb E[X_t] = 0
\end{equation*}

and 
\begin{equation*}
	\mathrm {Var}[X_t] = te^{-2t}
\end{equation*}

by solving ODEs for $\mathbb E[X_t]$ and $\mathbb E[X_t^2]$. \\

{\bf Solution 5}. In integral form our process $X_t$ is given by the SDE
\begin{equation*}
	X_t = X_0 - \int^t_0 X_u \,du + \int^t_0 e^{-u}\,dB_u
\end{equation*}

Taking the expectation 
\begin{align*}
	\mathbb E[X_t] &= \mathbb E\Big[0- \int^t_0 X_u \,du + \int^t_0 e^{-u}\,dB_u\Big] \\
	&= - \mathbb E\Big[\int^t_0 X_u \,du\Big] + \mathbb E\Big[\int^t_0 e^{-u}\,dB_u\Big] \quad \text{(by linearity)} \\
	&= - \mathbb E\Big[\int^t_0 X_u \,du\Big] 
\end{align*}

\indent Where the final line was achieved by realizing that $e^{-u}$ is $\mathcal F_t$-measurable and thereby applying the theorem verified in Problem 1. Moving on, we quickly verify that Fubini's theorem is not inappropriate here

\begin{align*}
	\mathbb E\Big[\int^t_0 X_u\,du \Big] &= \int_\Omega \int^t_0 X_u(\omega) \,du\,d\mathbb P(\omega) \\
	&= \int^t_0 \int_\Omega X_u(\omega) \,d\mathbb P(\omega)\,du \\
	&= \int^t_0 \mathbb E[X_u]\,du
\end{align*}

And so
\begin{align*}
	\mathbb E[X_t] &= -\mathbb E\Big[\int^t_0 X_u \,du\Big] \\
	&= -\int^t_0 \mathbb E[X_u]\,du
\end{align*}

Now we see that we have a natural ODE in $X_t$. Letting $\phi(t) = \mathbb E[X_t]$
\begin{align*}
	\phi(t) &= -\int^t_0 \phi(u)\,du \\
	\implies d\phi(t) &= -\phi(t)\,dt \\
	\implies \frac{d\phi(t)}{\phi(t)} &= -\,dt \\
	\implies \ln \phi(t) &= -t + c \\
	\implies \phi(t) &= e^{-t + c} \\
	&= Ce^{-t}
\end{align*}

Using our initial condition $X_0 = 0 = \mathbb E[X_0]$
\begin{align*}
	0 &= Ce^{-t} \\
	\implies C = 0
\end{align*}

and so we're left with the conclusion
\begin{equation*}
	\phi(t) = \mathbb E[X_t] = 0
\end{equation*}

as desired. Now, working on $\mathbb E[X_t^2]$, we lean on It\^{o}'s formula noting that since we are interested in $X_t^2$ we should consider $f(x) = x^2$. From It\^{o}'s formula
\begin{equation*}
	X_t^2 = X_0 + \int^t_0 2X_t\,dX_t + \frac{1}{2}\int^t_0 2\,d\langle X_{(\cdot)}\rangle_t
\end{equation*}

Since we were given $X_t$ in the question we quickly compute $dX_t$ and $d\langle X_{(\cdot)}\rangle_t$
\begin{align*}
	dX_t &= -X_t\,dt + e^{-t}\,dB_t \\
	d\langle X_{(\cdot)}\rangle_t &= \big(e^{-t}\big)^2\,dt
\end{align*}

Thus,
\begin{align*}
	X_t^2 &= X_0 + \int^t_0 2X_u\,dX_u + \frac{1}{2}\int^t_0 2\,d\langle X_{(\cdot)}\rangle_u \\
	&= 0 + \int^t_0 2X_u\Big[-X_u\,du + e^{-u}\,dB_u\Big] + \int^t_0 \big(e^{-u}\big)^2\,du \\
	&= - \int^t_0 2X_u^2\,du + 2\int^t_0 e^{-u}X_u\,dB_u + \int^t_0 e^{-2u}\,du \\
	&= - \int^t_0 2X_u^2\,du + 0 + \int^t_0 e^{-2u}\,du \quad \text{(from Problem 1)} \\
	&= \int^t_0 e^{-2u} - 2X^2_u \,du
\end{align*}

From this we see a natural ODE in $X_t^2$. Letting $\psi(t) = X_t^2$ 
\begin{align*}
	\psi(t) &= \int^t_0 e^{-2u} - 2\psi(u)\,du \\
	d\psi(t) &= \big(e^{-2u} - 2\psi(t)\big)\,dt \\
	\psi'(t) &= e^{-2u} - 2\psi(t)
\end{align*}

and so we go about solving our ODE in the typical manner
\begin{align*}
	e^{2t}\psi'(t) + 2e^{2t}\psi(t) &= 1 \\
	\frac{d}{dt}\Big(e^{2t}\psi(t)\Big) &= 1 \\
	\int \frac{d}{dt}e^{2t}\psi(t)\,dt &= \int 1\,dt \\
	e^{2t}\psi(t) &= t + C \\
	\psi(t) &= te^{-2t} + Ce^{-2t}
\end{align*}

Using our initial condition $X_0 = 0 \iff X_0^2 = 0 = \psi(0)$
\begin{align*}
	\psi(0) = 0 &= 0\cdot e^{-2\cdot0} + Ce^{-2\cdot 0} \\
	\implies C &= 0
\end{align*}

Thus we conclude with
\begin{align*}
	\psi(t) = X_t^2 &= te^{-2t} \\
	\therefore \quad \mathbb E[X_t^2] = \mathbb E[te^{-2t}] &= te^{-2t} 
\end{align*}

%=========Problem 6===========%
{\bf Problem 6}. Recall that stochastic integrals
\begin{equation*}
	\int^T_0 H_u\,dB_u
\end{equation*}

are martingales provided that the integrand $H$ is adapted and satisfies some technical (integrability) conditions. Using I\^{o}'s formula find a process $X_t$ such that
\begin{equation}
	B^3_t - X_t
\end{equation}

is a martingale.

{\bf Solution 6}. With $f(t,x,y) = x^3 - y$ we take our derivatives
\begin{align*}
	&f_t(t,x,y) = 0 \\
	&f_x(t,x,y) = 3x^2 \quad f_{xx}(t,x,y) = 6x \\
	&f_y(t,x,y) = -1 \quad f_{yy}(t,x,y) = 0 \\
	&f_{xy}(t,x,y) = 0
\end{align*} 

So, by It\^{o}'s formula evaluating $x = B_t$ and $y = X_t$, we have
\begin{align*}
	B_t^3 - X_t &= B^3_0 - X_0 + \int^t_0 3B_u^2\,dB_u + \int^t_0 (-1)\,dX_u + \frac{1}{2}\int^t_0 6B_u\,d\langle B_{(\cdot)}\rangle_u \\
	&= B^3_0 - X_0 + 3\int^t_0 B_u^2\,dB_u - \int^t_0 \,dX_u + 3\int^t_0 B_u\,du
\end{align*}

where the last line was achieved from recognizing that the quadratic variation of Brownian motion $\langle B_{(\cdot)}\rangle_t = dt$. Notice that without $-X_t$ we would be left with $B_t^3 = B^3_0 + 3\int^t_0 B_u^2\,dB_u + 3\int^t_0 B_u\,du$, showing us that $B_t^3$ is indeed not a martingale since it has a drift term $3\int^t_0 B_u\,du$.

%=========Problem 7===========%
{\bf Problem 7}.































\end{document}