% --------------------------------------------------------------
% This is all preamble stuff that you don't have to worry about.
% Head down to where it says "Start here"
% --------------------------------------------------------------
 
\documentclass[12pt]{article}
 
\usepackage[margin=1in]{geometry} 
\usepackage{amsmath,amsthm,amssymb,mathtools}
\usepackage{dsfont} % for indicator function \mathds 1
\usepackage{tikz,pgf,pgfplots}
\usepackage{enumerate} 
\usepackage[multiple]{footmisc} % for an adjascent footnote
\usepackage{graphicx,float} % figures
\usepackage{csvsimple,longtable,booktabs} % load csv as a table
\usepackage{listings,color} % for code snippets

\newenvironment{theorem}[2][Theorem:]{\begin{trivlist} %% Theorem Environment
\item[\hskip \labelsep {\bfseries #1}\hskip \labelsep {\bfseries #2.}]}{\end{trivlist}}

\newtheorem{definition}{Definition}
\let\olddefinition\definition
\renewcommand{\definition}{\olddefinition\normalfont}
\newtheorem{lemma}{Lemma}
\let\oldlemma\lemma
\renewcommand{\lemma}{\oldlemma\normalfont}
\newtheorem{proposition}{Proposition}
\let\oldproposition\proposition
\renewcommand{\proposition}{\oldproposition\normalfont}
\newtheorem{corollary}{Corollary}
\let\oldcorollary\corollary
\renewcommand{\corollary}{\oldcorollary\normalfont}

\newcommand\norm[1]{\left\lVert#1\right\rVert} % \norm command 

%%% PLOTTING PARAMETERS
\pgfmathsetseed{1952} % for Brownian Motion plotting
\newcommand{\Emmett}[5]{% points, advance, rand factor, options, end label
\draw[#4] (0,0)
\foreach \x in {1,...,#1}
{   -- ++(#2,rand*#3)
}
node[right] {#5};
}


\pgfplotsset{every axis/.append style={},
    cmhplot/.style={mark=none,line width=1pt,->},
    soldot/.style={only marks,mark=*},
    holdot/.style={fill=white,only marks,mark=*},
}

\tikzset{>=stealth}
%%%

%% set noindent by default and define indent to be the standard indent length
\newlength\tindent
\setlength{\tindent}{\parindent}
\setlength{\parindent}{0pt}
\renewcommand{\indent}{\hspace*{\tindent}}

\begin{document}
 
% --------------------------------------------------------------
%                         Start here
% --------------------------------------------------------------
 
\title{Mathematical \& Computational Finance II\\Lecture Notes}
\author{Continuous Time Finance}
\date{October 8 2015 \\ Last update: \today{}}
\maketitle

% SECTION: 
\section{Market Processes in Continuous Time}

\indent Recall that we were discussing finance \& option pricing in continuous time. Our modelling framework was a probability space $(\Omega,\mathcal F,\mathbb P)$ with filtration $\{\mathcal F_t\}_{t\geq0}$ generated by Brownian motion $B_t$ where $\mathbb P$ was the ``real world''\footnote{I want to elaborate more on what this means.}~ probability space. \\

We had a money market account $S^0_t$ such that
\begin{equation*}
	dS^0_t = rS^0_tdt, \quad S^0_0 = 1
\end{equation*}

with solution
\begin{equation*}
	S^0_t = e^{rt}
\end{equation*}

and a risky asset $S^1_t$ whose SDE
\begin{equation*}
	dS^1_t = \mu S^1_t\,dt + \sigma S^1_t\,dB_t, \quad S^1_0 = s_0
\end{equation*}

has solution\footnote{See October 1. The easiest way to see this is by starting with the solution \& applying It\^{o}'s formula to get the SDE.}
\begin{equation*}
	S^1_t = s_0e^{(\mu - \frac{1}{2}\sigma^2)t + \sigma B_t}
\end{equation*}

\indent Our wealth process $V_t$ associated with a portfolio $H = (H^0_t,H^1_t)$ of $S^0_t, S^1_t$, with $H^i_t$ denoting the quantity of asset $S^i$ at time $t$, was
\begin{equation*}
	V_t(H) = H^0_tS^0_t + H^1_tS^1_t
\end{equation*}

and we say $V_t$ is self financing if
\begin{align*}
	dV_t &= H^0_t\,dS^0_t + H^1_t\,dS^1_t \\
	&= H^0_trS^0_t\,dt + H^1_t\big[\mu S^1_t\,dt + \sigma S^1_t\,dB_t\big] \\
	&= H^0_t\big(rS^0_t\big)\,dt + H^1_t\big(\mu S^1_t\big)\,dt + H^1_t(\sigma S^1_t)\,dB_t
\end{align*}

\indent That is, movements in wealth come strictly from movements in the asset prices (i.e. no wealth movements from net injections/withdrawals of capital). Our discounted price processes are
\begin{align*}
	\overline{S}^1_t &= e^{-rt}S^1_t \\
	\overline{V}_t &= e^{-rt}V_t
\end{align*}

\indent We can apply It\^{o}'s formula on $f(t,x) = e^{-rt}x$ to find the SDEs that these processes satisfy. First taking our derivatives,
\begin{align*}
	f_t(t,x) &= -re^{-rt}x \\
	f_x(t,x) &= e^{-rt} \\
	f_{xx}(t,x) &= 0
\end{align*}

So for $\overline{S}^1_t$ we have
\begin{align*}
	\overline{S}^1_t = f(t,S^1_t) &= f(0,S^1_0) + \int^t_0 f_u(u,S^1_u)\,du + \int^t_0 f_x(u,S^1_u)\,dS^1_u + \frac{1}{2}\int^t_0 f_{xx}(u,S^1_u)\,d\langle S^1_{(\cdot)}\rangle_u \\
	&= e^{-r(0)}s_0 + \int^t_0 -re^{-ru}S^1_u\,du + \int^t_0 e^{-ru}\,dS^1_u + 0 \\
	&= 1 - r\int^t_0 e^{-ru}S^1_u\,du + \int^t_0 e^{-ru}\,dS^1_u \\
	&= 1 - r\int^t_0 e^{-ru}S^1_u\,du + \int^t_0 e^{-ru}\big[\mu S^1_u\,du + \sigma S^1_u\,dB_u \big] \\
	&= 1 - r\int^t_0 e^{-ru}S^1_u\,du  + \mu\int^t_0 e^{-ru}S^1_u\,du + \sigma\int^t_0 e^{-ru}S^1_u\,dB_u \\
	&= 1 + (\mu-r)\int^t_u e^{-ru}S^1_u\,du + \sigma\int^t_0 e^{-ru}S^1_u\,dB_u \\
	\implies d\overline{S}^1_t &= (\mu - r)\overline{S}^1_u\,dt + \sigma\overline{S}^1_t\,dB_t
\end{align*}

and for $\overline{V}_t$ we have
\begin{align*}
	\overline{V}_t = f(t,V_t) &= f(0,V_0) + \int^t_0 f_u(u,V_u)\,du + \int^t_0 f_x(u,V_u)\,dV_u + \frac{1}{2}\int^t_0 f_{xx}(u,V_u)\,d\langle V_{(\cdot)}\rangle_u \\
	&= e^{-r(0)}v_0 + \int^t_0 -re^{-ru}V_u\,du \int^t_0 e^{-ru}\,dV_u + 0 \\
	&= v_0 - r\int^t_0 e^{-ru}V_u\,du + \int^t_0 e^{-ru}\big[H^0_u\big(rS^0_u\big)\,du + H^1_u\big(\mu S^1_u\big)\,du + H^1_u(\sigma S^1_u)\,dB_u] \\
	&= v_0 - r\int^t_0 e^{-ru}V_u\,du + \int^t_0 e^{-ru}H^0_u\big(rS^0_u\big)\,du + \int^t_0 e^{-ru}H^1_u\big(\mu S^1_u\big)\,du \\
	&\hphantom{{}={v_0 - \int^t_0 e^{-ru}V_u\,du + \int^t_0 e^{-ru}H^0_u\big(rS^0_u\big)\,du + \int^t_0}} + \int^t_0 e^{-ru}H^1_u\big(\sigma S^1_u\big)\,dB_u \\
	&= v_0 - r\int^t_0 e^{-ru}\big(H^0_uS^0_u + H^1_uS^1_u\big)\,du + r\int^t_0 e^{-ru}H^0_uS^0_u\,du \\
	&\hphantom{{}={v_0 - r\int^t_0 e^{-ru}\big(H^0_uS^0_u + H^1_uS^1_u}} + \Big[\mu\int^t_0 e^{-ru}H^1_u S^1_u\,du + \sigma\int^t_0 e^{-ru}H^1_u S^1_u\,dB_u \Big] \\
	&= v_0 + r \Big[\int^t_0 e^{-ru}\Big(H^0_uS^0_u - H^0_uS^0_u - H^1_uS^1_u\Big)\,du\Big] \\
	&\hphantom{{}={v_0 - r\int^t_0 e^{-ru}\big(H^0_uS^0_u + H^1_uS^1_u}} + \Big[\mu\int^t_0 e^{-ru}H^1_u S^1_u\,du + \sigma\int^t_0 e^{-ru}H^1_u S^1_u\,dB_u \Big] \\
	&= v_0 - r\int^t_0 e^{-ru}H^1_uS^1_u\,du + \mu\int^t_0 e^{-ru}H^1_u S^1_u\,du + \sigma\int^t_0 e^{-ru}H^1_u S^1_u\,dB_u \\
	&= v_0 + (\mu - r) \int^t_0 e^{-ru}H^1_uS^1_u\,du + \sigma\int^t_0 e^{-ru}H^1_u S^1_u\,dB_u \\
	\implies d\overline{V}_t &= (\mu - r)H^1_t\overline{S}^1_t\,dt + \sigma H^1_t\overline{S}^1_t\,dB_t \\
	\iff d\overline{V}_t &= H^1_t\big[(\mu - r)\overline{S}^1_t\,dt + \sigma\overline{S}^1_t\,dB_t\big]
\end{align*}

\indent Note that our discounted risky asset and wealth processes have the term ($\mu - r$) appearing. Intuitively we can think of this as the risk premium for the risky asset. \\

\section{Probability Measures}

\indent Our goal is to be able to select a portfolio process whose payoff is equal to that of European contingent claim with payoff $h_T \in \mathcal F_T$ at time $T$. We should note that our ``real world'' measure $\mathbb P$ assigns probabilities to different states of the world (as do all measures) -- and these states in turn affect the value process $V_t$ (i.e. not necessarily a martingale). We say that these states and corresponding probabilities are a reflection of investors' beliefs. However, under $\mathbb P$ it's not usually possible to value $V_t$ as a discounted sum of independent cash flows since $V_t$ is not a martingale. \\

\indent So, we want to be able to construct a different probability measure $\mathbb Q$ (i.e. it assigns probabilities in a manner different than $\mathbb P$) under which our price process {\em is} a martingale. We call this measure $\mathbb Q$ the ``risk neutral'' measure.\footnote{``The idea behind the name ``risk neutral'' is that we may price securities as if we are indifferent to any volatility in the dividend stream or price process.''. That is, we are guaranteed that a replicating portfolio will always be the same value as our contingent claim.}~The key insight is that with the right choice of $\mathbb Q$ we not only have a price process that is now a martingale, but also expectations with respect to $\mathbb Q$ that are identical to those under $\mathbb P$ (i.e. the real world prices). \\

\indent Using risk neutral measure $\mathbb Q$ we are able to price things in the following way: Suppose that we have an easy no-arbitrage argument allowing us to pin down the value of our contingent claim. For example, if $V_t$ is the price process for a European call on asset with price $S^1_t$ with exercise date $T$ and strike price $K$, then $h_T = (S^1_T - K)^+$. In this case, the martingale property with respect to $\mathbb Q$ buys us

\begin{align*}
	V_t &= \mathbb E_{\mathbb Q}[e^{-r(T - t)}h_T] \quad \text{(by the martingale property)} \\
	&= \mathbb E_{\mathbb Q}[e^{-r(T - t)}(S^1_T - K)^+]
\end{align*}

\indent Now that we've seen the advantages of moving into a risk neutral framework we are faced with the obvious two questions
\begin{enumerate}
	\item How do we construct $\mathbb Q$?
	\item What does the path of our risky asset $S^1_t$ look like now that we've gone from $\mathbb P$ to $\mathbb Q$?
\end{enumerate}

\indent Girsanov's Theorem provides an answer to these, but itself relies on other results from stochastic calculus.

\begin{theorem}{L\'{e}vy's Theorem} {\em (Used to prove Girsanov's Theorem)} Suppose $(W_t)_{t\geq0}$ is a stochastic process on a probability space $(\Omega,\mathcal F,\mathbb P)$ and that $\{\mathcal F_t\}_{t\geq0}$ is the filtration generated by $W$. If
\begin{enumerate} 
	\item $W_t$ is continuous ($\mathbb P$ a.s.)
	\item $W$ is a $(\mathcal F_t,\mathbb P)$-martingale
	\item $W_t^2 - t$ is a $(\mathcal F_t,\mathbb P)$-martingale
\end{enumerate}

then $W_t$ is a standard Brownian motion.

\begin{proof} {\em Proof largely omitted, but here's a few notes.} \\

\indent Note that we've seen that the converse of this theorem is true since we define Brownian motion to have these properties, but it is not immediately obvious that this should hold going the other direction. \\

\indent The way that you prove L\'{e}vy's Theorem is by considering increments of $W$ and showing that the conditional\footnote{I think it's a good bet that we've got to use the tower property of conditional expectation: Condition and take out what's known.}~characteristic function is equal to a standard normal random variable. That is
\begin{align*}
	\mathbb E[e^{iu(W_t - W_s)}|\mathcal F_s] &= \cdots \\
	\vdots \\
	&= e^{-\frac{1}{2}u^2(t-s)} \\
	&= \mathbb E[e^{iu(W_t - W_s)}]
\end{align*}
\end{proof}
\end{theorem}

\indent The reason we need L\'{e}vy's Theorem is because we want to do something called a ``change of measure'' which relies on Girsanov's Theorem (discussed later) which itself relies on L\'{e}vy's Theorem. Eventually we will go into a risk neutral measure, but it's not trivial how to get there from our real world measure $\mathbb P$. \\

\subsection{Switching Probability Measures}

\indent The goal is to be able to enter a measure such that the discounted European contingent claim price process {\em is} a martingale. It turns out that (with some technical requirements) a price process $V_t$ for a derivative on $S_t$ avoids arbitrage opportunities only if a risk-neutral measure for the price process of the underlying $S$ is also a risk-neutral measure for $V_t$. So we see that we want to construct a risk-neutral measure for $S$. However, to do so we need to be careful which drift process, say $\Theta_t$, we select. \\

\indent On the probability space $(\Omega, \mathcal F,\mathbb P)$ with $(B_t)_{t\geq0}$ Brownian motion being a $(\mathcal F,\mathbb P)$-martingale, we consider the adapted process $(\Theta_t)_{t\geq0}$ and define\footnote{I think we call $\Lambda_t$ an ``exponential martingale''.}
\begin{equation*}
	\Lambda_t = e^{-\int^t_0\Theta_u\,dB_u - \frac{1}{2}\int^t_0 \Theta_u^2\,du } \\
\end{equation*}

\indent Letting $Z_t = -\int^t_0\Theta_u\,dB_u - \frac{1}{2}\int^t_0 \Theta_u^2\,du$ we get $\Lambda_t = e^{Z_t}$ and note that $\Lambda_t$ is a SDE for $Z_t$ with $f(t,x) = e^x$. From It\^{o}'s formula we know
\begin{align*}
	\Lambda_t = f(t,Z_t) &= f(0,0) + \int^t_0 f_u(u)\,du + \int^t_0 f_x(Z_u)\,dZ_u + \frac{1}{2}\int^t_0 f_{xx}(Z_u)\,d\langle Z_{(\cdot)}\rangle_u \\
	&= Z_0 + \int^t_0 e^{Z_u} \,dZ_u + \frac{1}{2}\int^t_0 e^{Z_u} \,d\langle Z_{(\cdot)}\rangle_u \\
	&= 1 + \int^t_0 e^{Z_u}\big[-\Theta_u\,dB_u - \frac{1}{2}\Theta_u^2\,du\big] + \frac{1}{2}\int^t_0 e^{Z_u} \,d\langle Z_{(\cdot)}\rangle_u \\
	&= 1 - \int^t_0 e^{Z_u}\Theta_u\,dB_u -  \frac{1}{2}\int^t_0 e^{Z_u} \Theta_u^2\,du + \frac{1}{2}\int^t_0 e^{Z_u} \,d\langle Z_{(\cdot)}\rangle_u \\
\end{align*}

Fortunately we know what the quadratic variation of an It\^{o} process is\footnote{See October 1, Proposition 2.}~, 
\begin{align*}
	\langle Z_{(\cdot)}\rangle_t &= \Big\langle \int^{(\cdot)}_0 -\Theta_u\,dB_u + \frac{1}{2}\int^{(\cdot)}_0 -\Theta^2_u\,du \Big\rangle_t \\
	&=  \Big\langle \int^{(\cdot)}_0 -\Theta_u\,dB_u \Big\rangle_t \\
	&= \int^t_0 \Theta_u^2\,du \\
	\implies d\langle Z_{(\cdot)}\rangle_t &= \Theta_t^2\,dt
\end{align*}

So
\begin{align*}
	\Lambda_t &= 1 - \int^t_0 e^{Z_u}\Theta_u\,dB_u - \frac{1}{2} \int^t_0 e^{Z_u}\Theta_u^2\,du + \frac{1}{2}\int^t_0 e^{Z_u} \,d\langle Z_{(\cdot)}\rangle_u \\
	&= 1 - \int^t_0 e^{Z_u}\Theta_u\,dB_u -  \frac{1}{2}\int^t_0 e^{Z_u} \Theta_u\,du + \frac{1}{2}\int^t_0 e^{Z_u}\Theta_u^2\,du \\
	&= 1 - \int^t_0 e^{Z_u}\Theta_u\,dB_u
\end{align*}

And so we end up with the SDE
\begin{align*}
	\Lambda_t &= 1 - \int^t_0\Lambda_u\Theta_u\,dB_u \\
	\implies d\Lambda_t &= -\Lambda_t\Theta_t\,dB_t
\end{align*}

We will propose $\Lambda_t$ as a candidate density but we should first be careful:
\begin{definition} A \underline{probability density function} on $(\Omega, \mathcal F,\mathbb P)$ is a $\mathcal F_t$-measurable random variable $\phi$ such that
\begin{enumerate}
	\item $\phi(\omega) > 0$ a.s.
	\item $\int_\Omega \phi(\omega)\,d\mathbb P(\omega) = \mathbb E[\phi] = 1$
\end{enumerate}
\end{definition}

Suppose $\Lambda_t$ is a martingale on $[0,T]$, so
\begin{equation*}
	\mathbb E[\Lambda_t] = \mathbb E[\Lambda_t|\mathcal F_0] = \Lambda_0 = 1
\end{equation*}

So $\Lambda_t$ is indeed a density and we will use it as a candidate as a probability density function for our proposed probability measure. We define a new probability measure $\mathbb P^\Theta$ on $(\Omega,\mathcal F_T)$ by
\begin{align*}
	\mathbb P^\Theta(A) &= \int_A \Lambda_T(\omega)\,d\mathbb P(\omega) \\
	&= \mathbb E[\mathds 1_A\Lambda_T] \quad \forall~A \in \mathcal F_T
\end{align*}

We can write (see Radon-Nikodym density/derivative)
\begin{equation*}
	\Lambda_T = \frac{d\mathbb P^\Theta}{d\mathbb P}\Bigg|_{\mathcal F_T}
\end{equation*}

\indent This is telling us how much probability weight $\mathbb P^\Theta$ is assigning to the states of the world relative to $\mathbb P$. It is capturing the adjustment we ought to be making to the probabilities given by $\mathbb P$. As an analogue, the density of the normal distribution tells you how much weight is assigned under the normal distribution to a given small interval of the real line, while the Radon-Nikodym density/derivative is telling us the weight assigned under $\mathbb P^\Theta$ to a small portion of the state space under $\mathbb P$. \\

The key is that $\Lambda_T$ is a (true) martingale if
\begin{equation*}
	\mathbb E_{\mathbb P}\Big[e^{-\frac{1}{2}\int^T_0\Theta_u^2\,du}\Big] < \infty
\end{equation*}

But under our new measure $\mathbb P^\Theta$ the Brownian motion $(B_t)_{t\geq0}$ is no longer a Brownian motion.\footnote{I'm not exactly sure what part we introduced the drift parameters.}

\begin{lemma} For a process $(X_t)_{t\geq0}$, $X_t\Lambda_t$ is a martingale under $\mathbb P$ if and only if $X_t$ is a martingale under $\mathbb P^\Theta$. \\

We will prove this in one direction.
\begin{proof} Suppose $X_t\Lambda_t$ is a $\mathbb P$ martingale. We want to show that
\begin{equation*}
	\mathbb E_{\mathbb P^\Theta}[X_t|\mathcal F_s] = X_s \quad \forall~0\leq s\leq t
\end{equation*}

\indent That is, $X_t$ is a martingale in our new measure. Essentially, the whole proof relies on a deep understanding of the definition of conditional expectation. From the definition,
\begin{equation*}
	\int_A \mathbb E_{\mathbb P^\Theta}[X_t|\mathcal F_s]\,d\mathbb P = \int_A X_s\,d\mathbb P^\Theta \quad \forall~A\in \mathcal F_s
\end{equation*}

Using the formalism\footnote{That is, this isn't fundamentally true but we use it as notation.}~$\Lambda_T\,d\mathbb P = d\mathbb P^\Theta$ we get
\begin{equation*}
	\int_A X_s\,d\mathbb P^\Theta = \int_A X_s \frac{d\mathbb P^\Theta}{d\mathbb P}\,d\mathbb P = \int_A X_s\Lambda_T\,d\mathbb P
\end{equation*}

Leaning on the definition of conditional expectation
\begin{align*}
	\int_A X_s\Lambda_T\,d\mathbb P &= \int_A\mathbb E_{\mathbb P}[X_s\Lambda_T|\mathcal F_s]\,d\mathbb P \\
	&= \int_A X_s \mathbb E_{\mathbb P}[\Lambda_T|\mathcal F_s]\,d\mathbb P \quad \text{(taking out what is known)} \\
	&= \int_A X_s\Lambda_s\,d\mathbb P_s \quad \text{(since $\Lambda_T$ is a $\mathbb P$ martingale)} \\
	&= \int_A \mathbb E_{\mathbb P}[X_t\Lambda_t|\mathcal F_s]\,d\mathbb P \quad \text{(by definition of conditional expectation)}\\
	&= \int_A X_t\Lambda_t\,d\mathbb P \quad \text{(by definition of conditional expectation)} \\
	\\
&\text{{\em ``A lot of these steps may seem useless, but they're really not.''}} \\
	\\
	&= \int_A X_t\mathbb E_{\mathbb P}[\Lambda_T|\mathcal F_t]\,d\mathbb P \\
	&= \int_A X_t\Lambda_T\,d\mathbb P \\
	&= \int_A X_t \frac{d\mathbb P^\Theta}{d\mathbb P}\,d\mathbb P \\
	&= \int_A X_t\,d\mathbb P^\Theta \\
	&= \int_A \mathbb E_{\mathbb P^\Theta}[X_t|\mathcal F_s]\,d\mathbb P^\Theta = \mathbb E_{\mathbb P^\Theta}[X_t|\mathcal F_s] = X_s
\end{align*}
\end{proof}
\end{lemma}

The point is that from this we get Girsanov's Theorem.
\begin{theorem}{Girsanov's Theorem} If $\Theta$ is adapted and 
\begin{equation*}
	\int^T_0 \Theta^2_u\,du < \infty
\end{equation*}

and
\begin{equation*}
	\Lambda_t = e^{-\int^t_0 \Theta_u\,dB_u - \frac{1}{2}\int^t_0\Theta^2_u\,du}
\end{equation*}

is a (true) martingale then the process $W$ defined by the SDE
\begin{align*}
	W_t &= B_t + \int^t_0\Theta_u\,du \\
	dW_t &= dB_t + \Theta_t\,dt
\end{align*}

is a standard Brownian motion $(\mathcal F_t,\mathbb P^\Theta)$
\begin{proof} ({\em Stated without proof, but done using L\'{e}vy's Theorem})
\end{proof}
\end{theorem}

\indent So, Girsanov's Theorem tells us that $W_t$ progresses as the sum of a Brownian motion under $\mathbb P$ and some process $\Theta_t$ (related to the Radon-Nikodym derivative characterizing $\mathbb P^\Theta$). We therefore want to choose a process $\Theta_t$ so that the path of $W_t$ with respect to $\mathbb P^\Theta$ cancels out the real world drift of the discounted process $\overline{S}^1_t$, leaving us with a pure Brownian motion, with respect to our new measure, to model the underlying asset.

\subsection{An Important Example}

Consider $\Theta_t$ is constant $\Theta \in \mathbb R$, so
\begin{equation*}
	W_t = B_t + \int^t_0 \Theta\,du = B_t + \Theta t
\end{equation*}

Under $\mathbb P$ we have
\begin{align*}
	B_T&\sim N(0,T) \\
	W_T&\sim N(\Theta t, T)
\end{align*}

\indent We introduce $\mathbb P^\Theta$ to remove the drift $\Theta t$ from the process $W_t$. This is a very important concept in mathematical finance since when we remove drift our process becomes a martingale (provided it otherwise satisfied the other criteria).

\subsection{Constructing the Risk Neutral Measure}

\begin{definition} If we set $\Theta = \frac{\mu - r}{\sigma}$ then the measure $\mathbb P^\Theta$ (from now on denoted $\mathbb Q$) is called the \underline{risk neutral} or \underline{martingale} measure.
\end{definition}

Setting $\Theta = \frac{\mu - r}{\sigma}$ we have
\begin{align*}
	W_t &= B_t + \Big(\frac{\mu - r}{\sigma}\Big)t \\
	\implies B_t &= W_t - \Big(\frac{\mu - r}{\sigma}\Big)t \\
	\implies dB_t &= dW_t - \Big(\frac{\mu - r}{\sigma}\Big)dt
\end{align*}

Using $S^1_t$ to solve for $dS^1_t$ under $\mathbb Q$ we have
\begin{align*}
	dS^1_t &= \mu S^1_t\,dt + \sigma S^1_t\,dB_t \\
	&= \mu S^1_t\,dt + \sigma S^1_t\Big[dW_t - \Big(\frac{\mu - r}{\sigma}\Big)dt\Big] \\
	&= \mu S^1_t\,dt + \sigma S^1_t\,dW_t - (\mu - r)S^1_t\,dt \\
	&= rS^1_t\,dt + \sigma S^1_t\,dW_t \quad \text{(remember that $W_t$ is Brownian under $\mathbb Q$)}
\end{align*}

\indent Notice that our drift is the risk-free rate. This aligns with the binomial model since the expected returns in this model is precisely the risk-free rate. Furthermore,
\begin{align*}
	d\overline{S}^1_t &= (\mu - r)\overline{S}^1_t\,dt + \sigma\overline{S}^1_t\,dB_t \\
	&= (\mu - r)\overline{S}^1_t\,dt + \sigma\overline{S}^1_t\Big[dW_t - \Big(\frac{\mu - r}{\sigma}\Big)dt\Big] \\
	&= (\mu - r)\overline{S}^1_t\,dt + \sigma\overline{S}^1_t\,dW_t - (\mu - r)\overline{S}^1_t\,dt \\
	&= \sigma\overline{S}^1_t\,dW_t
\end{align*}

\indent Notice that we have confirmed that we have now achieved our goal set above: Select a process $\Theta_t$ such that the discounted asset process $\overline{S}^1_t$ becomes Brownian. Additionally,
\begin{align*}
	d\overline{V}_t &= H^1_t\big[(\mu - r)\overline{S}^1_t\,dt + \sigma\overline{S}^1_t\,dB_t\big] \\
	&= H^1_t\big[(\mu - r)\overline{S}^1_t\,dt + \sigma\overline{S}^1_t\big( dW_t - \Big[\frac{\mu - r}{\sigma}\Big]dt \big)\big] \\
	&= H^1_t\big[(\mu - r)\overline{S}^1_t\,dt + \sigma\overline{S}^1_t\,dW_t - (\mu - r)\overline{S}^1_t\,dt\big] \\
	&= H^1_t\sigma\overline{S}^1_t\,dW_t
\end{align*}

We say $(\mu - r)$ is the risk premium and $\frac{\mu - r}{\sigma}$ the market price of risk.

\begin{definition} A \underline{martingale measure} is a probability measure $\mathbb Q$ that makes all discounted price processes martingales.
\end{definition}

Under $\mathbb Q$ the expected return is\footnote{Note that this is just notation since differentials are pretty handwavy.}
\begin{equation*}
	\mathbb E_{\mathbb Q}\Big[\frac{dS^1_t}{S_t}\Big] = r \,dt \\
\end{equation*}

\indent This means that the expected return is precisely the risk free rate, or the rate in our money market account. So, under $\mathbb Q$ we've essentially removed risk. How does this help us? Consider hedging \& replication.

\section{Option Pricing: The Replicating Portfolio}

Consider 
\begin{align*}
	M_t &= \int^t_0 \sigma H^1_u \overline{S}^1_u\,dW_u \quad \text{and} \\
	\overline{V}_t(H) &= \overline{V}_0 + M_t
\end{align*}

\indent In general these are local martingales.\footnote{``I'm not going to tell you what this means since it's unimportant for our purposes.''}\footnote{``There's actually a bunch of mathematics behind this but we'll just assume that it's a typical martingale.''} Because these are martingales we have something called a martingale representation.

\begin{theorem}{Martingale Representation Theorem/It\^{o} Representation Theorem\footnotemark}\footnotetext{The Martingale Representation Theorem is a generalized result and is sometimes referred to as the It\^{o} Representation Theorem when discussion Brownian Motion.} Suppose $(M_t,\mathcal F_t)_{0\leq t\leq T}$ is a square integrable martingale and $\mathcal F_t$ is the filtration generated by the Brownian motion $W_t$. Then, there exists an adapted process $(H_t)_{0\leq t \leq T}$ such that
\begin{enumerate}
	\item $\mathbb E[\int^T_0 H^2_u\,du] < \infty$
	\item $M_t = M_0 + \int^t_0 H_u\,dB_u \quad$ (this item is the key point)
\end{enumerate}

\begin{proof} {\em Stated without proof}
\end{proof}
\end{theorem}

Suppose that for a contingent claim $h_T$ we define the stochastic process $(N_t)_{t\geq 0}$ by
\begin{equation*}
	N_t = \mathbb E_{\mathbb Q}[e^{-r(T-t)}h_T|\mathcal F_t] \quad 0 \leq t \leq T
\end{equation*}

Then $N_t$ is a martingale with respect to $(\mathcal F_t, \mathbb Q)$. Why? Use the tower property of conditional expectation to find out! \\

So, by the Martingale Representation Theorem there exists a process $\gamma_t$ such that 
\begin{align*}
	N_t &= N_0 + \int^t_0 \gamma_s\,dW_s \\
	dN_t &= \gamma_t\,dW_t
\end{align*}

\indent That is, $\gamma_t$ is the process by which $N_t$'s movement follows\footnote{This is my interpretation.}, where $N_t$ is the discounted value of the contingent claim at time $t$. Note however that while we are guaranteed that $\gamma_t$ exists we aren't given what $\gamma_t$ is. Regardless, the MRT delivered us the existence of $\gamma_t$ to build our portfolio process. Take
\begin{align*}
	H^1_t &= \frac{\gamma_te^{rt}}{\sigma S^1_t} \\
	H^0_t &= N_t - \frac{\gamma_t}{\sigma}
\end{align*}

and consider the strategy $H^* = (H^0,H^1)$. Given this strategy we have to prove that
\begin{lemma} $H^*$ is self financing and $N_t = \overline{V}_t(H^*) = e^{-rt}V_t(H^*)$. \\

To check that $N_t = \overline{V}_t = e^{-rt}V_t(H^*)$ we can just plug in $H^*$ into our wealth process 
\begin{align*}
	\overline{V}_t(H^*) = e^{-rt}V_t(H^*) &= e^{-rt}\big[H^0_tS^0_t + H^1_tS^1_t\big] \\
	&= e^{-rt}\big[\big(N_t - \frac{\gamma_t}{\sigma}\big)S^0_t + \frac{\gamma_te^{rt}}{\sigma S^1_t}S^1_t\big] \\
	&= \big(N_t - \frac{\gamma_t}{\sigma}\big)e^{-rt}S^0_t + \frac{\gamma_t}{\sigma} \\
	&= N_t - \frac{\gamma_t}{\sigma} + \frac{\gamma_t}{\sigma} \quad \text{(since $e^{-rt}S^0_t = 1$)} \\
	&= N_t
\end{align*}

\indent So, we have $N_t = \overline{V}_t \iff N_0 + \int^t_0\gamma_u\,dW_u = \overline{V}_0 + \int^t_0 \sigma H^1_u\overline{S}^1_u\,dW_u$. It should be clear that the integrands $\gamma_t$ and $\sigma H^1_t\overline{S}^1_t$ are the same, but lets check using our proposal for $H^1_t$
\begin{equation*}
	H^1_t = \frac{\gamma_te^{rt}}{\sigma S^1_t} =  \frac{\sigma H^1_t \overline{S}^1_t e^{rt}}{\sigma S^1_t} =  \frac{\sigma H^1_t S^1_t}{\sigma S^1_t} = H^1_t \quad \text{as desired}
\end{equation*}

To check whether $H^*$ is self financing we must see that it satisfies the self financing condition
\begin{align*}
	dV_t(H) &= H^0_t\,dS^0_t + H^1_t\,dS^1_t \\
	&= H^0_trS^0_t\,dt + H^1_t\big[\mu S^1_t\,dt + \sigma S^1_t\,dB_t\big] \\
\end{align*}

That is, movements in wealth come strictly from movements in asset prices. So,
\begin{align*}
	V_t(H^*) = e^{rt}\overline{V}_t(H^*) &= e^{rt}\big(\overline{V}_0 + \int^t_0\sigma H^1_u\overline{S}^1_u\,dW_u \big) \\
	&= \overline{V_0} + e^{rt}\int^t_0\sigma H^1_u\overline{S}^1_u\,dW_u \\
\end{align*}

Applying It\^{o}'s formula with $V_t(H^*) = f(t,\overline{V}_t(H^*)) = e^{rt}\overline{V}_t(H^*) \equiv e^{rt}x$
\begin{align*}
	f(t,x) = V_t(H^*) &= V_0 + \int^t_0 re^{ru}\overline{V}_u(H^*)\,du + \int^t_0 e^{ru}\,d\overline{V}_u \\
	\implies dV_t(H^*) &= re^{rt}\overline{V}_t(H^*)\,dt + e^{rt}\,d\overline{V}_t \\
	&= re^{rt}\Big(e^{-rt}\big[H^0_tS^0_t + H^1_tS^1_t\big]\Big)\,dt + e^{rt}\big[\gamma_t\,dW_t\big] \\
	&= r\big[H^0_tS^0_t + H^1_tS^1_t\big]\,dt + e^{rt}\big[\sigma H^1_t S^1_t e^{-rt}\big]\,dW_t \\
	&\hphantom{{}={r\big[H^0_tS^0_t + H^1_tS^1_t\big]\,dt}} \text{(since the integrands of $\overline{V}_t$ and $N_t$ are identical)} \\
	&= H^0_trS^0_t\,dt + H^1_trS^1_t\,dt + H^1_t\sigma S^1_t,dW_t \\
	&= H^0_trS^0_t\,dt + H^1_t\big[rS^1_t\,dt + \sigma S^1_t\,dW_t \big] \quad \text{as desired}
\end{align*}


\indent We see that we get a SDE identical to the self financing portfolio and by existence \& uniqueness we know its the right one to satisfy our needs. Additionally, notice that we have $\mu = r$ in our self financing portfolio using strategy $H^*$, making the expected return precisely the riskless rate which: In line with our expected return under the martingale measure $\mathbb Q$.
\end{lemma}

Finally, note
\begin{align*}
	N_t = \overline{V}_t(H^*) &= \mathbb E_{\mathbb Q}[e^{-r(T-t)}h_T|\mathcal F_t] \quad 0 \leq t \leq T \\
	\implies \overline{V}_0(H^*) = V_0(H^*) &= \mathbb E_{\mathbb Q}[e^{-rT}h_T|\mathcal F_0] \\
	&= \mathbb E_{\mathbb Q}[e^{-rT}h_T|\mathcal F_0] \\
	&= e^{-rT}h_T \quad \text{(I think this is intuitive, but I'd like more)} \\
	\implies V_T(H^*) = e^{rT}\overline{V}_0(H^*) &= e^{rT}e^{-rT}h_T = h_T
\end{align*}

That is, $H^*$ replicates the payoff of the European contingent claim and we require
\begin{equation*}
	V_0(H^*) = \mathbb E_{\mathbb Q}[e^{-rT}h_T]
\end{equation*}

initial capital to hedge.

\subsection{The Minimal Hedge}

By absence of arbitrage type arguments\footnote{Left as an exercise, but maybe it's straightforward since you can consider cases where its $\leq$ and $\geq$.}~we have that
\begin{equation*}
	V_0(H^*) = \mathbb E_{\mathbb Q}[e^{-rT}h_T]
\end{equation*}

is the rational price at time 0 for the European contingent claim $h_T$. Thus if $\Phi$ is any other self financing portfolio process with initial capital $x$ then
\begin{equation*}
	V_T(\Phi) \geq V_T(H^*) = h_T
\end{equation*}

Thus $H^*$ is the minimal hedge.

\subsection{Determining the Process $\gamma_t$}

\indent Recall that we are guaranteed that the process $\gamma_t$ exists but we aren't given a recipe how to find out what it is. Thankfully we have ways for doing so. With our future model (Black-Scholes) we can figure out $\gamma_t$ which the MRT does not explicitly deliver but which our hedging depends heavily on.







\end{document}