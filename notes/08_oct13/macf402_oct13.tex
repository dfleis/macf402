% --------------------------------------------------------------
% This is all preamble stuff that you don't have to worry about.
% Head down to where it says "Start here"
% --------------------------------------------------------------
 
\documentclass[12pt]{article}
 
\usepackage[margin=1in]{geometry} 
\usepackage{amsmath,amsthm,amssymb,mathtools}
\usepackage{dsfont} % for indicator function \mathds 1
\usepackage{tikz,pgf,pgfplots}
\usepackage{enumerate} 
\usepackage[multiple]{footmisc} % for an adjascent footnote
\usepackage{graphicx,float} % figures
\usepackage{csvsimple,longtable,booktabs} % load csv as a table
\usepackage{listings,color} % for code snippets

\newenvironment{theorem}[2][Theorem:]{\begin{trivlist} %% Theorem Environment
\item[\hskip \labelsep {\bfseries #1}\hskip \labelsep {\bfseries #2.}]}{\end{trivlist}}

\newtheorem{definition}{Definition}
\let\olddefinition\definition
\renewcommand{\definition}{\olddefinition\normalfont}
\newtheorem{lemma}{Lemma}
\let\oldlemma\lemma
\renewcommand{\lemma}{\oldlemma\normalfont}
\newtheorem{proposition}{Proposition}
\let\oldproposition\proposition
\renewcommand{\proposition}{\oldproposition\normalfont}
\newtheorem{corollary}{Corollary}
\let\oldcorollary\corollary
\renewcommand{\corollary}{\oldcorollary\normalfont}

\newcommand\norm[1]{\left\lVert#1\right\rVert} % \norm command 

%%% PLOTTING PARAMETERS
\pgfmathsetseed{1952} % for Brownian Motion plotting
\newcommand{\Emmett}[5]{% points, advance, rand factor, options, end label
\draw[#4] (0,0)
\foreach \x in {1,...,#1}
{   -- ++(#2,rand*#3)
}
node[right] {#5};
}


\pgfplotsset{every axis/.append style={},
    cmhplot/.style={mark=none,line width=1pt,->},
    soldot/.style={only marks,mark=*},
    holdot/.style={fill=white,only marks,mark=*},
}

\tikzset{>=stealth}
%%%

%% set noindent by default and define indent to be the standard indent length
\newlength\tindent
\setlength{\tindent}{\parindent}
\setlength{\parindent}{0pt}
\renewcommand{\indent}{\hspace*{\tindent}}

\begin{document}
 
% --------------------------------------------------------------
%                         Start here
% --------------------------------------------------------------
 
\title{Mathematical \& Computational Finance II\\Lecture Notes}
\author{The Black-Scholes World}
\date{October 13 2015 \\ Last update: \today{}}
\maketitle

% SECTION: 
\section{The Minimal Hedge}

\indent Last time we had that if $f_T$ (previously denoted $h_T$) is the payoff of a European contingent claim which may be exercised at time $T$ we need that
\begin{align*}
	f_T &\in \mathcal F_T \quad \text{(i.e. $f_T$ is measurable at time $T$)} \\
	\mathbb E_{\mathbb P}[e^{-rT}f_T] &< \infty \quad \text{(i.e. the discounted payoff is integrable wrt $\mathbb P$)}
\end{align*}

Then we can say that the rational/no-arbitrage price is
\begin{equation*}
	C(T,f_T) = \mathbb E_{\mathbb Q}[e^{-rT}f_T]
\end{equation*}

where $\mathbb Q$ is our risk neutral measure. To construct $\mathbb Q$ we take $\Theta = \frac{\mu - r}{\sigma}$ (under the Black-Scholes model) and have $\mathbb Q$ be defined by, for $0 \leq t \leq T$,
\begin{align*}
	\Lambda_t &= e^{-\int^t_0\Theta_u\,dB_u - \frac{1}{2}\int^t_0\Theta^2_u\,du} \quad \text{and} \\
	\mathbb Q(A) &= \int_A \Lambda\,d\mathbb P \quad \forall~A\in\mathcal F_T \quad \text{and} \\
	W_t &= B_t + \int^t_0\Theta_u\,du
\end{align*}

and we have the result where $W_t$ is a Brownian motion on $(\Omega,\mathcal F,\mathbb Q)$. We also had found a minimal hedge/portfolio process $H^* = (H^0,H^1)$ given by
\begin{align*}
	H^1_t &= \frac{\gamma_t}{\sigma}\frac{e^{rt}}{S^1_t} \\
	H^0_t &= N_t - e^{rt}S^1_tH^1_t
\end{align*}

where $N_t = \mathbb E_{\mathbb Q}[e^{-rt}f_T|\mathcal F_t]$ and $\gamma_t$ is known to exist by the Martingale Representation Theorem such that $N_t = N_0 + \int^t_0 \gamma_s\,dW_s$. We had determined that the value $C(T,f_T)$ is the amount of initial capital needed to replicate the option payoff using the portfolio process $H^*$.

\section{Derivation of the Black-Scholes Price in the Risk Neutral Framework}

On $(\Omega,\mathcal F,\mathbb P)$ (the real world space) we have
\begin{align*}
	dS^0_t &= rS^0_t\,dt \\
	dS^1_t &= \mu S^0_t\,dt + \sigma S^1_t\,dB_t
\end{align*}

and on the risk neutral space $(\Omega,\mathcal F,\mathbb Q)$ we have
\begin{align*}
	dS^0_t &= rS^0\,dt \\
	dS^1_t &= rS^1_t\,dt + \sigma S^1_t\,dW_t
\end{align*}

Using It\^{o}'s formula we can show that the solution to the SDE for $S^1_t$, in the risk neutral space, is
\begin{equation*}
	S^1_t = S^1_0 e^{(r-\frac{1}{2}\sigma^2)t + \sigma W_t}
\end{equation*}

\begin{theorem}{Risk Neutral Pricing Theorem} In the Black-Scholes model any option defined by a nonnegative $\mathcal F_T$-measurable random variable, say $f_T$ which is square integrable under $\mathbb Q$ (and thus $\mathbb P$), is replicable. The value at time $t \in [0,T]$ of any replicating portfolio is
\begin{equation*}
	V_t = \mathbb E_{\mathbb Q}[e^{-r(T-t)}f_T|\mathcal F_t]
\end{equation*}

\begin{proof} We could sketch this proof but we've basically already done it using the Martingale Representation Theorem, etc... The only modification is that now we have to deal with $(T - t)$ appearing.
\end{proof}
\end{theorem}

If we assume $f_T = f(S_T)$ then we have
\begin{equation*}
	V_t = \mathbb E_{\mathbb Q}[e^{-r(T-t)}f(S_T) | \mathcal F_T]
\end{equation*}

Note that
\begin{align*}
	\frac{S^1_T}{S^1_t} &= \frac{S^1_0\exp{\big[(r - \frac{1}{2}\sigma^2)T - \sigma W_T}\big]}{{S^1_0\exp\big[(r - \frac{1}{2}\sigma^2)t - \sigma W_t)\big]}} \\
	\iff S^1_T &= S^1_t e^{(r - \frac{1}{2}\sigma^2)(T - t) + \sigma(W_T - W_t)} \\
	&= \mathbb E_{\mathbb Q}\Big[ e^{-r(T - t)}f(S^1_te^{(r - \frac{1}{2}\sigma^2)(T - t) + \sigma(W_T - W_t)}) \Big| \mathcal F_t\Big]
\end{align*}

Noting that $W_T - W_t$ is independent of our filtration and $S^1_t$ is $\mathcal F_t$ measurable.

\begin{theorem}{``FACT''} On a probability space $(\Omega,\mathcal G,\mathbb P)$ let $X$ and $Y$ be random variables and $\mathcal A$ a sub-$\sigma$-algebra of $\mathcal G$. Suppose $X$ is $\mathcal A$-measurable and $Y$ is independent of $\mathcal A$, then for any bounded measurable function $f(X,Y)$ define some other function $\phi(X)$ as
\begin{equation*}
	\phi(X) = \mathbb E[f(x,Y)] \quad \forall~x\in\mathbb R
\end{equation*}

then we have
\begin{equation*}
	\mathbb E[f(x,Y)|\mathcal A] = \phi(X) =  \mathbb E[f(x,Y)]
\end{equation*}

Basically, this is telling us that if we take a point $x$ (since $X$ is $\mathcal A$ measurable we have that $x$ is known) and we can just compute the ordinary expectation as desired.

\end{theorem}

Moving on, with 
\begin{equation*}
	F(t,x) = \mathbb E_{\mathbb Q}[e^{-r(T - t)}f\big(x,^{(r - \frac{1}{2}\sigma^2)(T - t) + \sigma(W_T - W_t)}\big)]
\end{equation*}

we have that 
\begin{equation*}
	V_t = F(t,S^1_t)
\end{equation*}

and $\sigma(W_T - W_t)$ is just normally distributed with mean 0 and variance $\sigma^2(T-t)$ so this is easily computable, and using the standard normal density we have
\begin{equation*}
	F(t,x) = e^{-r(T-t)}\int^\infty_{-\infty} f\big(xe^{(r - \frac{1}{2}\sigma^2)(T - t) + \sigma\sqrt{(T - t)}z}\big) \cdot \frac{1}{\sqrt{2\pi}}e^{-\frac{1}{2}z^2}\,dz
\end{equation*}

\indent From this we have successfully reduced the Black-Scholes option pricing problem to a relatively simple integration problem.

\begin{lemma} For constant $k > 0$ we have that the conditional probability under the measure $\mathbb Q$ 
\begin{equation*}
	\mathbb E_{\mathbb Q}[\mathds 1_{S_T > k} | \mathcal F_t] = 
	\Phi\Big[\frac{\log(\frac{S_t}{K}) + (T - t)(r - \frac{\sigma^2}{2})}
		{\sigma\sqrt{T - t}}\Big]
\end{equation*}

where $\Phi(x)$ is the normal CDF and the conditional $\mathcal F_t$ means that we are given all the information available to about the system up to time $t$ (i.e. the price). We prove this using basic calculus \& some integration tricks:

\begin{proof} Let
\begin{equation*}
	d_2 = \frac{\log(\frac{S_t}{K}) + (T - t)(r - \frac{\sigma^2}{2})}
		{\sigma\sqrt{T - t}}
\end{equation*}

so that
\begin{equation*}
	V_0 = e^{-rt}\Phi[d_2]
\end{equation*} \\

Note that since $S_T \equiv S_te^{(r - \frac{1}{2}\sigma^2)(T - t) + \sigma(W_T - W_t)}$ we have
\begin{align*}
	S_T > K \iff S_te^{(r - \frac{1}{2}\sigma^2)(T - t) + \sigma(W_T - W_t)} > K \\
	\implies \log(S_t) + (r - \frac{1}{2}\sigma^2)(T - t) + \sigma(W_T - W_t) > \log(K) \\
	\implies W_T - W_t > \frac{-\log(\frac{S_t}{K}) - (r - \frac{1}{2}\sigma^2)(T - t)}{\sigma}
\end{align*}

For brevity let $Y_t = \frac{\log(\frac{S_t}{K}) + (r - \frac{1}{2}\sigma^2)(T - t)}{\sigma}$ so that we have $W_T - W_t > -Y_t$, hence
\begin{equation*}
	\mathbb E_{\mathbb Q}[\mathds 1_{S_T > K}|\mathcal F_t] = \mathbb E_{\mathbb Q}[\mathds 1_{W_T - W_t > -Y_t} | \mathcal F_t]
\end{equation*}

\indent Since we have that $Y_t$ is $\mathcal F_t$-measurable and $(W_T - W_t)$ is independent of our filtration we may use our theorem above (``FACT'') so that
\begin{align*}
	\mathbb E_{\mathbb Q}[\mathds 1_{W_T - W_t > -Y_t} | \mathcal F_t] &= \mathbb E_{\mathbb Q}[\mathds 1_{W_T - W_t > -Y_t}] \\
	&= \frac{1}{\sqrt{2\pi(T-t)}} \int^\infty_{-Y_t} e^{-\frac{1}{2(T - t)}z^2}\,dz \\
	&= \frac{1}{\sqrt{2\pi(T-t)}} \int^{Y_t}_{-\infty} e^{-\frac{1}{2(T - t)}z^2}\,dz \\
\end{align*}

With the substitution
\begin{equation*}
	u(z) = \frac{z}{\sqrt{T - t}} \implies d[u(z)] = \frac{dz}{\sqrt{T - t}}
\end{equation*}

we have
\begin{align*}
	u(Y_t) &= \frac{\log(\frac{S_t}{K}) + (r - \frac{1}{2}\sigma^2)(T - t)}{\sigma\sqrt{T - t}} \equiv d_2\\
	\implies \frac{1}{\sqrt{2\pi(T-t)}} \int^{Y_t}_{-\infty} e^{-\frac{1}{2(T - t)}z^2}\,dz &=
	\frac{1}{\sqrt{2\pi(T-t)}} \int^{d_2}_{-\infty} e^{-\frac{1}{2(T - t)}[u\sqrt{T - t}]^2}\big[\sqrt{T - t}\,du\big] \\
	&= 	\frac{1}{\sqrt{2\pi}} \int^{d_2}_{-\infty} e^{-\frac{1}{2}u^2}\,du \\
	&= \Phi[d_2]
\end{align*}

as desired.
\end{proof}
\end{lemma}

\subsection{The Binary Option}

Consider the example of a binary option with payoff
\begin{equation*}
	h(S_T) =
	\begin{cases}
		1 & \text{if } S_T > K \\
		0 & \text{else}
	\end{cases}
\end{equation*}

\indent We say that this European-style contingent is a ``cash-or-nothing'' call option. We can find the correct price for this option using the lemma above. In this example we consider time $t = 0$,

\begin{align*}
	V_0 &= \mathbb E_{\mathbb Q}[e^{-rT}h(S_T)] \\
		&= e^{-rT}\mathbb E_{\mathbb Q}[h(S_T)] \quad \text{(since $e^{-rt}$ is known)} \\
		&= e^{-rT}\mathbb E_{\mathbb Q}[\mathds 1_{S_T > K}] \\
		&= e^{-rT}\Phi[d_2]
\end{align*}

with $d_2$ to be defined as earlier.

\subsection{The ``Asset-or-Nothing'' Option}

Consider the example of an option with payoff 
\begin{equation*}
	h(S_T) = 
	\begin{cases}
		S_T & \text{if } S_T > K \\
		0 & \text{else}
	\end{cases}
\end{equation*}

\indent We call this the ``asset-or-nothing'' call option. By our risk neutral pricing formula we have that, at time $t = 0$,
\begin{equation*}
	V^{AON}_0 = \mathbb E_{\mathbb Q}[ e^{-rT}S_T\cdot \mathds 1_{S_T > K}]
\end{equation*}

\indent This is similar to the ``all-or-nothing'' option but we now have the included $S_T$ term. We proceed by recall our substitution for $S_T = S_te^{(r - \frac{1}{2}\sigma^2)(T - t) + \sigma(W_T - W_t)}$ and evaluating at $t = 0$ to get
\begin{align*}
	V^{AON}_0 &= \mathbb E_{\mathbb Q}[e^{-rT}S_0e^{(r - \frac{1}{2}\sigma^2)(T - 0) + \sigma(W_T - W_0)}\cdot \mathds 1_{S_T > K}] \\
	&= \mathbb E_{\mathbb Q}[S_0e^{\frac{1}{2}\sigma^2T + \sigma W_T} \cdot \mathds 1_{S_T > K}] \\
\end{align*}

Again recalling our lemma we have
\begin{align*}
	V^{AON}_0 &= \mathbb E_{\mathbb Q}[S_0e^{\frac{1}{2}\sigma^2T + \sigma W_T} \cdot \mathds 1_{W_T > -Y_0}] \\
		&= \mathbb E_{\mathbb Q}[S_0e^{\frac{1}{2}\sigma^2T + \sigma W_T} \cdot \mathds 1_{-W_T < Y_0}] \quad \text{(by symmetry of $W_T\sim N(0, T)$)} \\
		&= \mathbb E_{\mathbb Q}[S_0e^{\frac{1}{2}\sigma^2T - \sigma(-W_T)} \cdot \mathds 1_{-W_T < Y_0}] \quad \text{(to make all our $W_T$ terms the same sign)} \\
		&= S_0e^{\frac{1}{2}\sigma^2T}\mathbb E_{\mathbb Q}[e^{-\sigma(-W_T)} \cdot \mathds 1_{-W_T < Y_0}] \quad \text{(taking out what is known)} \\
		&= S_0e^{\frac{1}{2}\sigma^2T}\frac{1}{\sqrt{2\pi T}} \int^{Y_0}_{\infty} e^{-\sigma z} e^{-\frac{1}{2T}z^2}\,dz
\end{align*}

From here we use the substitution
\begin{align*}
	&u = \frac{z}{\sqrt{T}} \iff z = u\sqrt{T} \implies dz = \sqrt{T}\,du \\
	\therefore~&u(Y_0) = d_2
\end{align*}

So we simplify our integral to
\begin{align*}
	V^{AON}_0 &= S_0e^{\frac{1}{2}\sigma^2T}\frac{1}{\sqrt{2\pi T}} \int^{Y_0}_{\infty} e^{-\sigma z} e^{-\frac{1}{2T}z^2}\,dz \\
	&= S_0e^{\frac{1}{2}\sigma^2T}\frac{1}{\sqrt{2\pi T}} \int^{d_2}_{\infty} e^{-\sigma u\sqrt{T}} e^{-\frac{1}{2T}(u\sqrt{T})^2}\big[\sqrt{T}\,du\big] \\
	&= S_0e^{\frac{1}{2}\sigma^2T}\frac{1}{\sqrt{2\pi}} \int^{d_2}_{\infty} e^{-\sigma u\sqrt{T}} e^{-\frac{1}{2}u^2}\,du \\
	&= S_0\frac{1}{\sqrt{2\pi}} \int^{d_2}_{\infty} e^{-\frac{1}{2}u^2 - \sigma u \sqrt{T} - \frac{1}{2}\sigma^2T}\,du
\end{align*}

We realize that our integrand is a perfect square (or complete the square to see this),
\begin{align*}
	-\frac{1}{2}u^2 - \sigma\sqrt{T}u  - \frac{1}{2}\sigma^2T &= -\frac{1}{2}\big(u^2 + 2\sigma\sqrt{T}u + \sigma^2T \big) \\
	&= -\frac{1}{2}\big(u + \sigma\sqrt{T})^2
\end{align*}

So our integral becomes
\begin{equation*}
	V^{AON}_0 = S_0 e^{-rT}\frac{1}{\sqrt{2\pi}}\int^{d_2}_{\infty} e^{-\frac{1}{2}(u + \sigma\sqrt{T})^2}\,du
\end{equation*}

Substituting again
\begin{align*}
	&v = u + \sigma\sqrt{T} \implies dv = du \\
	\therefore~&v(d_2) = d_2 + \sigma\sqrt{T} 
\end{align*}

Letting $d_1 = d_2 + \sigma\sqrt{T}$ we have
\begin{align*}
	V^{AON}_0 &= S_0\int^{d_2}_{\infty} e^{-\frac{1}{2}(u + \sigma\sqrt{T})^2}\,du \\
	&= S_0\int^{d_1}_{\infty} e^{-\frac{1}{2}v^2}\,dv \\
	&= S_0\Phi[d_1]
\end{align*}

Therefore we conclude that the correct price at time $t = 0$ for a European-style ``asset-or-nothing'' call option is
\begin{equation*}
	V^{AON}_O = S_0\Phi[d_1]
\end{equation*}

with $d_1$ defined as above.

\section{The European Call Option}

We claim that the Black-Scholes time $t = 0$ price of a European call option with payoff
\begin{equation*}
	h(S_T) = (S_T - K)^+
\end{equation*}

is
\begin{equation*}
	C_0 = S_0\Phi[d_1] - Ke^{-rT}\Phi[d_2]
\end{equation*}

\indent That is, we claim that the correct price is identical to a portfolio of 1 long ``asset-or-nothing'' call options and and K short ``cash-or-nothing'' call options. Where
\begin{align*}
	d_1 &= \frac{\log(\frac{S_0}{K}) + (r + \frac{1}{2}\sigma^2)T}{\sigma\sqrt{T}} \\
	d_2 &= d_1 - \sigma\sqrt{T} = \frac{\log(\frac{S_0}{K}) + (r - \frac{1}{2}\sigma^2)T}{\sigma\sqrt{T}}
\end{align*}

\begin{proof} By the risk neutral pricing argument we have that
\begin{align*}
	C_0 &= \mathbb E_{\mathbb Q}[e^{-rT}(S_T - K)^+] \\
		&= \mathbb E_{\mathbb Q}[e^{-rT}(S_T - K) \mathds 1_{S_T > K}] \\
		&= \mathbb E_{\mathbb Q}[e^{-rT} S_T \mathds 1_{S_T > K}] - \mathbb E_{\mathbb Q}[e^{-rT} K \mathds 1_{S_T > K}] \\
		&= \mathbb E_{\mathbb Q}[e^{-rT} S_T \mathds 1_{S_T > K}] - K\mathbb E_{\mathbb Q}[e^{-rT} \mathds 1_{S_T > K}] \\
		&= S_0\Phi[d_1] - Ke^{-rT}\Phi[d_2]
\end{align*}
\end{proof}

In generality if we're at time $t$, for $0 \leq t < T$, we have
\begin{equation*}
	C_t = S_t\Phi[d_1(t)] - Ke^{-r(T - t)}\Phi[d_2(t)]
\end{equation*}

where
\begin{align*}
	d_1(t) &= \frac{\log(\frac{S_t}{K}) + (r + \frac{1}{2}\sigma^2)(T - t)}{\sigma\sqrt{T - t}} \\
	d_2(t) &= d_1(t) - \sigma\sqrt{T - t} = \frac{\log(\frac{S_t}{K}) + (r - \frac{1}{2}\sigma^2)(T - t)}{\sigma\sqrt{T - t}}
\end{align*}

\subsection{The European Put Option}

\indent We could go through these steps again to derive a price for a European-style put option, but if we already have the price of a call option then its instead quicker to do so using put-call parity:
\begin{equation*}
	C - P = S_t - e^{-r(T - t)}K
\end{equation*}

So,
\begin{align*}
	P_t &= e^{-r(T - t)}K - S_t + \Big(S_t\Phi[d_1(t)] - Ke^{-r(T - t)}\Phi[d_2(t)]\Big) \\
		&= Ke^{-r(T - t)}\big(1 - \Phi[d_2(t)]\big) - S_t\big(1 - \Phi[d_1(t)]\big)
\end{align*}

By symmetry of the normal distribution, $1 - \Phi(x) = \Phi(-x)$, we have
\begin{equation*}
	P_t = Ke^{-r(T - t)}\Phi[-d_2(t)] - S_t\Phi[-d_1(t)]
\end{equation*}

To have done this rigorously we would need to prove put-call parity, but we'll call it sufficient for now.

\section{Hedging}

\indent When we were constructing the hedging process $H^* = (H^0, H^1)$ we figured out the ratios for each component
\begin{align*}
	H^1_t &= \frac{\gamma_te^{rt}}{\sigma S^1_t} \\
	H^0_t &= N_t - \frac{\gamma_t}{\sigma}
\end{align*}

but were left without the $\gamma_t$ guaranteed to exist by the Martingale Representation theorem. If we're actually going to do anything with this process we'll need to make this $\gamma_t$ component explicit. Fortunately we have that
\begin{equation*}
	V_t = \mathbb E_{\mathbb Q}\Big[e^{-r(T - t)}f(S^1_t\cdot\exp{\big[(r - \frac{1}{2}\sigma^2)(T - t) + \sigma(W_T - W_t)\big]} \Big| \mathcal F_t \Big]
\end{equation*}

By our ``FACT'' we know this is just a function of the asset price and time 
\begin{equation*}
	V_t = e^{-r(T - t)}F(T - t, S^1_t)
\end{equation*}

We can prove (it's not easy) that in this model $F$ is differentiable with respect to both $t$ and $x$, so, we may write
\begin{equation*}
	G(t,x) = F(T - t, e^{rt}x)
\end{equation*}

then we have
\begin{align*}
	V_te^{-rt} &= e^{-rT}G(t, e^{-rt}S^1_t) \\
	\overline{V}_t &= e^{-rT}G(t, \overline{S}^1_t)
\end{align*}

Applying It\^{o}'s formula gives us
\begin{equation*}
	\overline{V}_t = e^{-rT}\Bigg[G(0, \overline{S}^1_0) + \int^t_0 \frac{\partial}{\partial u} G(u, \overline{S}^1_u)\,du + \int^t_0 \frac{\partial}{\partial x} G(u, \overline{S}^1_u)\,d\overline{S}^1_u + \frac{1}{2}\int^t_0 \frac{\partial^2}{\partial x^2} G(u, \overline{S}^1_u)\,d\langle \overline{S}^1_{(\cdot)} \rangle_u \Bigg]
\end{equation*}

\indent Under the probability measure $\mathbb Q$ we know that $\overline{S}^1$ and $\overline{V}$ are martingales thus no drift term should appear in their SDEs. Therefore, since we know that
\begin{align*}
	d\overline{S}^1_t = \overline{S}^1_t\sigma\,dW_t \quad \text{(notice no drift in $dt$)} \\
	d\langle \overline{S}^1_{(\cdot)} \rangle_t = \big(\overline{S}^1_t\sigma\big)^2\,dt
\end{align*}

we must have that
\begin{align*}
	\overline{V}_t &= e^{-rT}\Bigg[G(0, \overline{S}^1_0) + \int^t_0 \frac{\partial}{\partial u} G(u, \overline{S}^1_u)\,du + \int^t_0 \frac{\partial}{\partial x} G(u, \overline{S}^1_u)\,d\overline{S}^1_u + \frac{1}{2}\int^t_0 \frac{\partial^2}{\partial x^2} G(u, \overline{S}^1_u)\,d\langle \overline{S}^1_{(\cdot)} \rangle_u \Bigg] \\
	&= e^{-rT}\Bigg[G(0, \overline{S}^1_0) + \int^t_0 \frac{\partial}{\partial u} G(u, \overline{S}^1_u)\,du + \int^t_0 \frac{\partial}{\partial x} G(u, \overline{S}^1_u)\overline{S}^1_u\sigma\,dW_u + \frac{1}{2}\int^t_0 \frac{\partial^2}{\partial x^2} G(u, \overline{S}^1_u)\big(\overline{S}^1_u\sigma\big)^2\,du \Bigg] \\
	&= e^{-rT}\Bigg[G(0, \overline{S}^1_0) + \int^t_0 \Bigg(\frac{\partial}{\partial u} G(u, \overline{S}^1_u) + \frac{1}{2}\frac{\partial^2}{\partial x^2} G(u, \overline{S}^1_u)(\overline{S}^1_u)^2\sigma^2\Bigg)\,du + \int^t_0 \frac{\partial}{\partial x} G(u, \overline{S}^1_u)\overline{S}^1_u\sigma\,dW_u\Bigg] \\
\end{align*}

has as its drift term $du$ be  equal to zero. That is, we must satisfy
\begin{equation*}
	\frac{\partial}{\partial u} G(u, \overline{S}^1_u) + \frac{1}{2}\frac{\partial^2}{\partial x^2} G(u, \overline{S}^1_u)(\overline{S}^1_u)^2\sigma^2 = 0
\end{equation*}

Using 
\begin{align*}
	G(T,x) &= F(0, e^{rT}x) = f(e^{rT},x) \\
	G(t,x) &= F(T - t, e^{rt}x)
\end{align*}

we apply the chain rule for partial derivatives on our equation above to obtain the partial differential equation
\begin{equation*}
	\frac{\partial F}{\partial u}(u, S^1_u) + rS^1_t\frac{\partial F}{\partial x}(u, S^1_u) + \frac{1}{2}\sigma^2(S^1_t)^2\frac{\partial^2 F}{\partial x^2}F(u, S^1_u) = 0
\end{equation*}













\end{document}